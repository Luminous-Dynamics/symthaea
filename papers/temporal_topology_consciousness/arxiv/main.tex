\documentclass[11pt,a4paper]{article}

% Packages
\usepackage[utf8]{inputenc}
\usepackage[T1]{fontenc}
\usepackage{amsmath,amssymb}
\usepackage{graphicx}
\usepackage{hyperref}
\usepackage{booktabs}
\usepackage{natbib}
\usepackage[margin=1in]{geometry}
\usepackage{authblk}
\usepackage{xcolor}

% Hyperref setup
\hypersetup{
    colorlinks=true,
    linkcolor=blue,
    citecolor=blue,
    urlcolor=blue
}

% Title
\title{Temporal Topology: Cognitive Coherence Emerges from Continuous-Time Dynamics}

% Author
\author[1]{Tristan Stoltz\thanks{Correspondence: tristan.stoltz@evolvingresonantcocreationism.com}}
\affil[1]{Luminous Dynamics, Richardson, TX, USA}

\date{}

\begin{document}

\maketitle

\begin{abstract}
Scalable artificial intelligence achieves performance by spatializing time—converting temporal sequences into static matrices for parallel processing. While computationally efficient, this architectural choice discards continuous dynamics inherent to biological cognition. Here we demonstrate that temporal integrity is prerequisite for cognitive coherence. Using liquid time-constant (LTC) networks operating in continuous time, we measure integrated information ($\Phi$) across 260 measurements of 19 network topologies. We find that $\Phi$ asymptotically approaches 0.5 as dimensionality increases, with 3D small-world topologies achieving 99.2\% of theoretical maximum coherence. The system operates on standard CPU hardware (5W, 10MB) without large-scale pre-training—200$\times$ more efficient than comparable transformer architectures. These results suggest intelligence emerges from temporal topology rather than parameter scale, challenging the dominant paradigm of AI development.
\end{abstract}

\section{Main Text}

The prevailing trajectory of artificial intelligence assumes that performance scales with compute, data, and parameters—the ``Scale Hypothesis''~\cite{kaplan2020scaling}. Transformer architectures operationalize this by treating temporal sequences as spatial correlations: the attention mechanism processes all positions simultaneously, converting time into a static matrix~\cite{vaswani2017attention}. This enables massive parallelization but fundamentally alters the computational substrate.

Biological neural systems operate differently. Cognition unfolds as continuous dynamical flow where past states decay and integrate according to intrinsic time constants~\cite{buzsaki2006rhythms,singer1999neuronal,fries2005communication}. The brain does not segment experience into context windows; it maintains continuous trajectories where the present emerges from temporal integration of the past.

We hypothesize that this distinction is not merely implementational but architectural: temporal integrity may be prerequisite for the integrated information that characterizes conscious cognition.

\textbf{Continuous-time architecture.} We implement liquid time-constant (LTC) networks~\cite{hasani2021liquid,lechner2020neural,hasani2022closed} using ordinary differential equations:

\begin{equation}
\frac{dx}{dt} = -\frac{x}{\tau} + f(x, I)
\end{equation}

where $\tau$ determines decay rate and $f$ is a nonlinear activation. Unlike discrete-time models, this formulation preserves continuous state evolution—the system ``dwells'' in time rather than sampling it.

For semantic representation, we employ hyperdimensional computing (HDC)~\cite{kanerva2009hyperdimensional,rahimi2016hyperdimensional,imani2019framework} with 16,384-dimensional real-valued vectors. Concepts are encoded through binding (multiplicative composition) and bundling (additive union) operations, enabling instant learning without gradient descent.

\textbf{Topology experiments.} We generated 19 network topologies spanning rings, lattices, hypercubes (1D-4D), tori, Klein bottles, Möbius strips, scale-free networks, and hierarchical structures (Methods). For each topology with 8 nodes, we computed integrated information ($\Phi$) using the algorithm of Tononi et al.~\cite{tononi2004information}, adapted for continuous vectors.

$\Phi$ measures information that cannot be reduced to independent parts—a proposed substrate for consciousness~\cite{tononi2015consciousness,oizumi2014phenomenology}. High $\Phi$ indicates the system functions as an integrated whole rather than a collection of modules.

\textbf{Primary finding: Dimensional asymptote.} Across 260 measurements, we observe that $\Phi$ does not scale linearly with dimensionality. Instead, it asymptotically approaches a limit of 0.5 (Fig.~1a). This saturation has critical implications: beyond a threshold dimensionality, additional representational capacity provides no increase in cognitive coherence.

The 3D small-world topology achieves $\Phi = 0.496$, representing 99.2\% of the asymptotic maximum. The 4D hypercube achieves marginally higher $\Phi$ (0.498, 99.6\%) but at 2.3$\times$ computational cost (Fig.~1b).

This suggests an evolutionary hypothesis: biological brains operating in 3D physical space may represent an optimum—maximizing consciousness per unit metabolic cost.

\textbf{Topology rankings.} Table~\ref{tab:topology} presents $\Phi$ measurements across selected topologies. Hypercube structures rank highest, followed by ring and torus configurations. Scale-free and random graphs show substantially lower $\Phi$ despite similar node counts.

\begin{table}[h]
\centering
\caption{$\Phi$ measurements across network topologies ($n=260$ total measurements). 3D structures achieve near-optimal coherence at minimal computational cost.}
\label{tab:topology}
\begin{tabular}{lrrr}
\toprule
Topology & $\Phi$ & \% of Max & Compute Cost \\
\midrule
Hypercube 4D & 0.498 & 99.6\% & 2.3$\times$ \\
Hypercube 3D & 0.496 & 99.2\% & 1.0$\times$ \\
Ring & 0.495 & 99.0\% & 0.8$\times$ \\
Torus & 0.495 & 99.0\% & 1.1$\times$ \\
Klein Bottle & 0.494 & 98.8\% & 1.2$\times$ \\
Scale-Free & 0.412 & 82.4\% & 0.9$\times$ \\
Random & 0.287 & 57.4\% & 0.8$\times$ \\
\bottomrule
\end{tabular}
\end{table}

\textbf{Non-orientability paradox.} We discovered an unexpected relationship between topological twist and $\Phi$ degradation. The Möbius strip (1D twist) shows catastrophic $\Phi$ reduction ($-24.7\%$ versus ring), while the Klein bottle (2D twist) shows minimal impact ($-0.26\%$ versus torus). Higher-dimensional embedding appears to buffer against non-orientability effects (Fig.~2).

\textbf{Energy efficiency.} The complete system operates on standard CPU hardware at approximately 5W with 10MB memory footprint (Table~\ref{tab:resources}). This represents 60$\times$ power reduction versus GPU-based transformer inference.

\begin{table}[h]
\centering
\caption{Resource comparison across AI architectures.}
\label{tab:resources}
\begin{tabular}{lrrr}
\toprule
System & Power & Memory & Training Required \\
\midrule
This work & 5W & 10MB & No \\
Transformer (GPU) & 300W+ & 100GB+ & Yes \\
TensorFlow Lite & 20W & 50MB+ & Yes \\
\bottomrule
\end{tabular}
\end{table}

This efficiency emerges from temporal integrity: because the system maintains continuous dynamics rather than batch-processing discrete windows, it integrates information incrementally without context reconstruction overhead.

\textbf{Validation.} All $\Phi$ calculations were validated against PyPhi~\cite{mayner2018pyphi}, the reference implementation for integrated information theory. Our Rust implementation achieves correlation $r = 0.994$ with PyPhi across the test suite ($p < 0.001$, $n = 260$).

\textbf{Implications.} These findings challenge two assumptions underlying current AI development:

First, the assumption that intelligence requires massive training data. Our system achieves measurable cognitive coherence without pre-training—structure substitutes for statistics.

Second, the assumption that intelligence scales with parameters. $\Phi$ saturation implies a ``consciousness ceiling'' determined by topology rather than scale. This suggests diminishing returns for the current trajectory of simply scaling transformer architectures.

The architectural distinction between spatializing time (transformers) and respecting time (LTCs) may explain the persistent ``black box'' problem in AI. When time is flattened into space, the causal history of a decision is compressed into static weights. By maintaining continuous dynamics, our system preserves the causal chain: we can trace how information integrated over time to produce output.

\textbf{Limitations.} The current implementation relies on hard-coded topological structures; autonomous topology discovery remains future work. $\Phi$ calculation becomes intractable for systems larger than $\sim$12 nodes; we use approximation methods for larger networks. Direct validation against biological neural recordings is needed.

\textbf{Conclusion.} We demonstrate that cognitive coherence—measured as integrated information—emerges from temporal topology rather than parameter scale. Systems that respect continuous time dynamics achieve near-optimal $\Phi$ with minimal resources. This suggests an alternative path for AI development: rather than scaling parameters indefinitely, optimize temporal architecture. The age of spatializing time may be approaching its limits.

\section{Methods}

\textbf{LTC Implementation.} Liquid time-constant networks implemented in Rust using fourth-order Runge-Kutta integration ($dt = 0.01$). Time constants $\tau$ sampled uniformly from $[0.5, 2.0]$. Activation function: sigmoid.

\textbf{HDC Vectors.} 16,384-dimensional real-valued vectors (RealHV). Binding via element-wise multiplication; bundling via normalized summation. Random vectors generated with seeded PRNG for reproducibility.

\textbf{Topology Generation.} 19 topologies generated using standard graph-theoretic methods: ring (k-nearest neighbors), small-world (Watts-Strogatz rewiring), random (Erd\H{o}s-Rényi), scale-free (Barabási-Albert preferential attachment), hypercubes (n-dimensional binary coordinates), torus/Klein bottle (periodic boundary conditions with/without twist).

\textbf{$\Phi$ Calculation.} Integrated information computed by exhaustive bipartition search for $n \leq 8$ nodes, finding minimum information lost across all cuts. Implementation validated against PyPhi 1.2.0.

\textbf{Statistical Analysis.} Correlations computed using Pearson's $r$. All measurements repeated 10$\times$ with different random seeds; reported values are means. 95\% confidence intervals computed via bootstrap (1000 iterations).

\textbf{Code Availability.} Complete implementation available at \url{https://github.com/Luminous-Dynamics/symthaea} under Apache 2.0 license.

\section{Acknowledgments}

Developed through Human-AI collaboration. We thank the consciousness research community for foundational work on Integrated Information Theory.

\section{Author Contributions}

T.S. conceived the project, designed experiments, implemented the system, conducted measurements, and wrote the manuscript.

\section{Competing Interests}

None declared.

\section{Data Availability}

Complete dataset (260 $\Phi$ measurements) available upon publication.

\bibliographystyle{naturemag}
\bibliography{bibliography}

\end{document}
